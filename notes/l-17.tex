\lecture{Lecture - 17\hfill 07 Oct 24, Mon}

\section{The Inclusion-Exclusion Principle}
The simplest possible manifestation of the inclusion exclusion Principle
is
$$ \lvert A \cap B \rvert = \lvert A \rvert + \lvert B \rvert - \lvert A \cap B \rvert$$
for any two finite substes $A$ and $B$ of a set $X.$

\begin{theorem}
	Let $A$ be a finite set, and $A_1, A_2, \dotsc A_k \subseteq A.$
	Then
	$$ \lvert A \setminus \bigcup_{j \in [k]} A_j \rvert
	= \sum_{J \subseteq[k]} (-1)^{ \lvert  J \rvert} \lvert A_J \rvert$$
	where $A_J = \bigcap_{j \in J} A_j,$ and $A_{\phi} = A.$
\end{theorem}

\begin{remark}
	We can think of $A_1, A_2, \dotsc, A_k$ as a bunch of bad sets
	which we want to exclude from $A.$
\end{remark}

\begin{proof}[My Proof]
	Let $S_a = \{ i \in [n] \, : \, a \in A_i \}$ for each $a$ in 
	$A.$ Let $B = \bigcup_{i \in [n]} A_i.$ We may write
\begin{align*}
	\lvert B \rvert% 
	={}& \sum_{a \in B} 1 \\
	={}& \sum_{a \in B} 1 - (1-1)^{ \lvert S_a \rvert} \\
	\intertext{(because $ \lvert S_a \rvert \not = 0$ for $a$
	in $B.$)}
	={}& \sum_{a \in B} \sum_{j=1}^{ \lvert S_a \rvert} (-1)^j
	\binom{ \lvert S_a \rvert}{j} \\
	={}& \sum_{a \in B} \sum_{j=1}^{ \lvert S_a \rvert} (-1)^j
\sum_{\substack{ J \subseteq [S_a]\\ \lvert J \rvert = j}} 1 \\
	\intertext{(the binomial coefficient is simply the number of ways of choosing $j$ sized subsets from $S_a.$)}
	={}& \sum_{a \in B} \sum_{j=1}^{ \lvert S_a \rvert}
\sum_{\substack{ J \subseteq [S_a]\\ \lvert J \rvert = j}} (-1)^{ \lvert J \rvert} \\
	={}& \sum_{a \in B} \sum_{J \subseteq [S_a]} (-1)^{ \lvert J \rvert} \\ 
	={}& \sum_{a \in b} \sum_{J \subseteq [n]} (-1)^{ \lvert J \rvert} \prod_{i \in J} \chi_i(a) \\
	\intertext{(because the product $\prod_{i \in J} \chi_i(a) $
		is $1$ if $a$ is in $\bigcap_{i \in J} A_i$ and
	$0$ otherwise)}
	={}& \sum_{J \subseteq [n]} (-1)^{ \lvert J \rvert}
	\sum_{a \in B} \chi_1(a) \cdots \chi_n(a) \\
	\intertext{(because finite double summations can be interchanged.)}
	={}& \sum_{J \subseteq [n]} (-1)^{ \lvert J \rvert}
	\lvert \bigcap_{j \in J} A_j \rvert .
\end{align*}
\end{proof}

\iffalse
\begin{proof}
	The right hand side of the equation may be rewritten as
\begin{align*}
	& \sum_{m=0}^{k} (-1)^m \sum_{\substack{J \subseteq [k] \\ \lvert J \rvert = m}} \lvert A_J \rvert \\
	={}& \sum_{m=0}^{k} \sum_{\substack{J \subseteq[k] \\
	\lvert J \rvert = m}} \chi_J(i)
\end{align*}
Fix $a$ in $A.$
Then
\begin{align*}
	={}& \sum_{\substack{ J \subseteq [k] } (-1)^{ \lvert J \rvert}}
		\chi_{A_J} (a) \\
	& \sum_{\substack{ J \subseteq [k]\\ a \in A_J }} \chi_{A_J} (a) \\
	={}& \sum_{J \subseteq S_i} (-1)^{ \lvert J \rvert},
\end{align*}
 where $S_i = \{ j \in [k] \,:\, a \in A_j \}.$
 If $S_i \not = \phi,$ then $ \sum_{J \subseteq [k]} (-1)^{ \lvert J \rvert} \chi_{J}(i) = (-1)^0 \chi_{A_\phi}(a) = 1.$
 If $S_i \not = \phi,$ then
 \begin{align*}
	& \sum_{J \subseteq [j]} (-1)^{ \lvert J \rvert } \chi_{A_J} (i)\\
	={}& \sum_{J \subseteq [S_i] } (-1)^{  }\\
	={}& \sum_{j=0}^{ \lvert S_i \rvert} (-1)^i \sum_{
	\substack{ J \subseteq [k] \\ \lvert J \rvert = j}} 1 \\
	={}& d
 \end{align*}
\end{proof}

\fi
