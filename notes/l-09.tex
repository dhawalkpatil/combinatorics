
\lecture{Lecture - 09\hfill18 Sep 24, Wed}

\begin{definition}[Permutation]
	Let $S$ be a set. A function $ \sigma \colon S \to S$ is said to be a
	permutation on $S$ if it is a bijection.
\end{definition}

The set $ \Omega (S) $ of all the permutations on $S$ is a group when equipped with the
composition operation.
That is, for any $\sigma, \tau, \mu$ in $ \Omega(S),$ we have
$$ \sigma \circ ( \tau \circ \mu) = (\sigma \circ \tau) \circ \mu.$$
In other words, the composition operation is associative.
There exists the identity permutation $ \iota $ in $ \Omega(S)$
satisfying $\iota \circ \sigma = \sigma = \sigma \circ \iota$
for every $\iota$ in $ \Omega(S).$

The point $a \in S$ is said to be a fixed point of $ \sigma $ if $ \sigma(a) = a.$

The cycle $(a_1, a_2, \dotsc, a_k)$ denotes a permutation defined by:
$$ \sigma(a_1) = a_2, \sigma(a_2) = a_3, \dotsc, \sigma(a _{k-1} ) = a_k, \sigma(a_k) = a_1$$
and 
$$ \sigma(n) = n$$
for all $t $ in $S$ such that $t \not \in \{a_1, a_2, \dotsc, a_k\}.$

If $\sigma$ and $ \tau$ are disjoint cycles, that is,
$ \sigma =  (a_1, a_2, \dotsc, a_k)$ and $\tau = (b_1, b_2, \dotsc, b_l) $
and $a_i \not = b_j$ for any $i = 1, \dotsc, k $ or $ l =1 ,\dotsc, l,$
then $ \sigma \circ \tau = \tau \circ \sigma.$

\begin{definition}[$S_n$]
	The set of permutations on the set $\left[ n \right] = \left\{ 1, 2, \dotsc, n \right\} $ is denoted by $S_n.$
\end{definition}

\begin{theorem}
	Any permutation $ \sigma $ in $S_n$ can be written as a product of disjoint cycles.
	Moreover, this decomposition is unique upto reordering of the disjoint cycles.
\end{theorem}

\begin{proof}
	Consider the permutation $\sigma$ in $S_n.$
Since the fixed points of $\sigma$ do not appear in the cyclic decomposition of $\sigma,$
we induct on $m,$ the number of points in $ \left[ n \right] $ which are not fixed by $ \sigma.$
In other words, $ m = \lvert \left\{ a \in \left[ n \right] \, : \, \sigma(a) \not = a \right\}  \rvert .$
If $ m = 2,$ then there are exactly two points $a_1, a_2$ which are not fixed by $ \sigma.$
Then $\sigma = (a_1, a_2).$

Induction hypothesis:
Suppose every permutation $\sigma$ in $S_n$ with at most $k-1$ non fixed points
is a product of disjoint cycles.
We proceed to prove that $\sigma$ in $S_n$ is a product of disjoint cycles
if it has $k$ non-fixed points.
Pick a point $a_0$ in $ \left[ n \right] $ which is not a fixed point of $\sigma.$
We then examine the elements $$a_1 = \sigma(a_0), a_2 = \sigma^2(a_0), \dotsc, a_k = \sigma^k(a).$$
As these are all non fixed points of $\sigma,$ at least two of them are the same 
because these are $k+1$ in number and there are only $m$ non fixed points of 
$\sigma.$ Suppose $\sigma^j(a_0) = a_j = a_l = \sigma^l(a_0).$
If $j=0$ and $l=m,$ then 
$$ ( a_0 \, \sigma(a_0) \, \dotsc \, \sigma^{m-1}(a_0) )$$
is an $k$\nobreakdash-cycle.
Otherwise $ \tau = (a_j, a _{j+1}, \dotsc, a _{l-1})$ is a cycle with $l-1-j$
non fixed points.
For $s = \sigma(a_i),$ where $i=j, j+1, \dotsc l-1,$ we have
$$ \tau^{-1} \circ \sigma (s) = \tau^{-1}(\sigma^{i+1}(a_0)) = \sigma^i(a_0)$$
so that $\tau^{-1} \circ \sigma$ leaves $a_j, a _{j+1}, \dotsc, a_{l-1}$ fixed.
$\tau$ also fixes all the points which are fixed by $ \sigma,$ so all 
the other $n - k$ points fixed by $\sigma$ are fixed by $\tau^{-1} \circ \sigma.$

Thus, it has $m - l + j < m $ non fixed points.
It can thus be expressed as a product of disjoint cycles as
$$ \tau^{-1} \circ \sigma = 
(a_{1\, 1} \, a_{1\, 2} \, \dotsc \, a_{1 \, k_1} )
(a_{2\, 1} \, a_{1\, 2} \, \dotsc \, a_{2 \, k_1} )
\cdots
(a_{l\, 1} \, a_{l\, 2} \, \dotsc \, a_{l \, k_1} )
.$$
$\tau$ is itself a cycle, so we have a cyclic decomposition for $\sigma$
as follows
$$ ( a_0 \, \sigma(a_0) \, \dotsc \, \sigma^{m-1}(a_0) )
(a_{1\, 1} \, a_{1\, 2} \, \dotsc \, a_{1 \, k_1} )
(a_{2\, 1} \, a_{1\, 2} \, \dotsc \, a_{2 \, k_1} )
\cdots
(a_{l\, 1} \, a_{l\, 2} \, \dotsc \, a_{l \, k_1} ) . $$
\end{proof}

\begin{proof}[Alternative proof]
Pick $a_1$ which is not a fixed point of $\sigma.$
If $\sigma^j(a) \not = a$ for $j = 1, \dotsc, m-1,$
then $\sigma = (a, \sigma(a), \sigma^2(a), \dotsc, \sigma^{m-1}(a) )$
is a cycle of length $m.$
If, otherwise, $\sigma^j(a) = a$ for some positive integer $j \leq m-1,$
we have $\sigma = \tau \circ \mu$ 
where $\tau(a) = \sigma(a)$ and $\tau^l(a) = \sigma^l(a)$
for $l = 1, 2, \dotsc, j-1.$ $\tau$
is a permutation which fixes all points other than those in
$T = \left\{ a, \sigma(a), \dotsc , \sigma^{j-1}(a) \right\},$
and $\mu$ is the restriction of $\sigma$ to 
$\left[ n \right]  \setminus T.$
That is,
$$\mu(s) = \begin{cases}
	s, \quad & \text{ if } s \in T\\
	\sigma(s), \quad & \text{ otherwise }
\end{cases} $$
We know that $\tau$ fixes all points not in $T.$
We want to show that $\mu$ fixes all points in $T.$
Note that we can express $a = \sigma^j(a).$
If $\mu(b) = \sigma^i(a)$ for some element $\sigma^i(a)$ in $T,$
then $\mu(b) = \sigma(b),$ meaning that $b = \sigma^{i-1}(a) $ is in $T.$
However, this is not possible because $\mu$ is chosen to fix all points
in $T.$ We need to prove that $\mu$ is a permutation.
Let $r,t$ be in $\left[ n \right] .$ If $r$ and $t$ are both 
in $T,$ then $ \mu(t) = t $ and $ \mu(r) = r.$ If neither $r$ nor
$t$ are in $T,$ then $ \mu(t) = \sigma(t) \not = \sigma(r) = \mu(r)$ 
because $\sigma$ is injective.
If both $r$ ant $t$ are not in $T,$
If one of $r, t$ is in $T,$ then exactly one of 
$\mu(r),$ $\mu(t)$ is in $T,$ so again they are not equal.
Thus, $\mu$ is injective.
Let $s$ be in $\left[ n \right].$ If $s = \sigma^i(a)$ for some positive integer $i \leq l,$
then $s $ is in $T$ and $s = \mu(s),$ otherwise $ s = \sigma(s') = \mu(s') $ for some $s'$ in
$ \left[ n \right] \setminus T$ because $ \sigma$ is surjective.
This shows that $ \mu$ is surjective.
Finally, we have that $ T = \left\{ a, \sigma(a), \dotsc, \sigma^{j-1}(a) \right\} $
has cardinality $j$ which is less than $m$ 
and so it leaves more than $n-m$ points fixed.
So, $\mu$ has at least $n-m+j$ points which are fixed.
Proceeding in this way we can express $\sigma$ as 
$$ \tau \circ \tau_2 \circ \cdots \circ \tau_k \circ \mu.$$
Here, $\mu$ has at least $n-m + j + 2(k-1)$ fixed points.
In finitely many steps, we will reach a point where
$\mu$ has at least $n-2$ fixed points. This would mean 
that $\mu$ itself is a cycle.
\end{proof}

\begin{theorem}[Decomposition into Transpositions]\label{thm:decomposition-into-transpositions}
	Any permutation $\sigma$ in $S_n$ can be written as a product of transpositions
	which may not be disjoint.
\end{theorem}

$$ 
\left( 
	\begin{array}{c}
		(a_{1,1} a_{1,2} \dotsc a_{1,n_1})\\
		(a_{2,1}  a_{2,2} \dotsc a_{2,n_2})\\
		\vdots \\
		(a_{k,1} a_{k,2} \dotsc a_{k,n_k}) 
	\end{array}
	\right)
	= \begin{array}{c}
		(a_{1,1} a_{1,2}) (a_{1,1} a_{1,3}) (a_{1,1} a_{1,4}) \cdots (a_{1,1} a_{1,n_1})\\
		(a_{2,1} a_{2,2}) (a_{2,1} a_{2,3}) (a_{2,1} a_{2,4}) \cdots (a_{2,1} a_{2,n_2}) \\
		 \vdots \\
		 (a_{k,1} a_{k,2}) (a_{k,1} a_{k,3}) (a_{k,1} a_{k,4}) \cdots (a_{k,1} a_{k,n_2}) \\
	\end{array}
	$$
