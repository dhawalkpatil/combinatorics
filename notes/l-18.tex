
\noindent
\emph{Lecture - 18\hfill 09 Oct 24, Wed}

\begin{definition}[Derangements]
	If $n$ in $\mathbb{N}$ is a positive integer and $\sigma \colon [n] \to [n]$ is a permutation of $[n],$ i.e., $\sigma \in S_n,$ such that
	$\sigma(i) \not = i,$ for any $i in $ $[n],$ then $\sigma$
	is said to be a derangement of $[n].$
\end{definition}
\begin{theorem}
	If $d_n$ is the number of derangements of $[n],$ then
	$$ d_n = \sum_{j=0}^{n}  (-1)^j \binom{n}{j} (n-j)!
	= \sum_{j=0}^{n} (-1)^j \frac{n!}{j!} $$
\end{theorem}

\begin{remark}
	$\sum_{j=0}^{n} \frac{(-1)^j}{j!} $ is an approximation for 
	$ \frac{1}{e}.$
\end{remark}

\begin{proof}
	Let $A = S_n$ and $A_i$ be the set of all permutations $\sigma$
	in $S_n$ such that $\sigma)i) = i.$ So, according to  the Inclusion Exclusion Principle, the number of permutations which do not fix any elements at all, $d_n,$ is
	\begin{align*}
		d_n = {}& \lvert A \setminus \bigcup_{i \in [n]} A_i \rvert \\
		={}& \sum_{J \subseteq [n]} (-1)^{ \lvert J \rvert} 
		\left( \lvert A_J \rvert \right) \\
		={}& \sum_{J \subseteq [n]} (-1)^{ \lvert J \rvert} 
		(n - \lvert J \rvert )!\\
		={}& \sum_{j=0}^{n} (-1)^j \binom{n}{j} (n-j)!
	\end{align*}
\end{proof}

This theorem has a nice probabilistic interpretation.
Suppose we have a large number $n$ of individuals in a room who put their 
umbrellas down at the door when they come in. If each person randomly picks up an umbrella while leaving, then the probability that no person gets their umbrella back is $ \frac{d_n}{n!}$ which tends to $ \frac{1}{e}$ as $n \to \infty.$

\begin{theorem}[Counting Surjective maps]
	The number of surjective maps from $[n]$ to $[k]$ is 
	$$ \sum_{j=0}^{k} (-1)^j \binom{k}{j} (k-j)^n.$$
\end{theorem}

\begin{proof}
Let $A_i = \left\{ f \colon [n] \to [k] \; \middle| \;
f(j) \not =  i \, \forall \, j \in [n] \right\}.$
So, $ \lvert A_i \rvert = (k - \lvert A_j \rvert )^n.$
By Inclusion Exclusion Principle, we get that the cardinality of $S$, 
the set of all surjective functions from $[n]$ to $[k].$
\begin{align*}
	\lvert S \rvert ={}& \sum_{J \subseteq [k]} (-1)^{ \lvert J \rvert} \lvert A_J \rvert \\
	={}& \sum_{j=0}^{n} (-1)^j \sum_{\substack{J \subseteq [k] \\
\lvert J \rvert = j }} (k - j)^n \\
	={}& \sum_{j=0}^{n} (-1)^j \binom{k}{j} (k-j)^n.
\end{align*}
\end{proof}


\section{Stirling numbers of the first kind}
Recall that we hada  factorization of permutations as a product of disjoint cucles (unique upto obvious invariances).
The unsigned Stirling number of the first kind 
are 
$$ \begin{bmatrix} n\\k \end{bmatrix} $$
which is the number of permutations of $[n]$ which can be written as a product of $k$ disjoint cycles, over positive integers $n,k$ such that $k \leq n .$
The \emph{Signed Stirling numbers of the first kind are} $$s(n,k) = (-1)^{n-k} \begin{bmatrix} n\\k \end{bmatrix} , $$
where $k \leq n$ are positive integers.

Note $s(n,n) = 1$ because only the identity permuatation on $[n]$ can be written as a product of $n$ disjoint cycles. Observe that
$$ \sum_{j=1}^{n}  \begin{bmatrix} n \\ k \end{bmatrix} 
= \lvert S_n \rvert = n!. $$

\begin{theorem}
	If $n \geq k$ are positive integers, then 
	$$ s(n+1, k) = - n \times s(n,k) + s(n, k-1). $$
\end{theorem}

\begin{proof}
Consider $$S_{n,k} = \left\{ \sigma \in S_n \; : \;
\sigma \text{ can be written as a product of }
k \text{ disjoint cycles } \right\}, $$
and write $S_{n+1, k} = A \cup B,$ where
$A = \left\{ \sigma \in S_{n+1, k}  \, : \, \sigma(n+1) = n+1 \right\} .$
We may write $ \lvert A \rvert = \begin{bmatrix} n \\ k-1 \end{bmatrix} = (-1)^{n-k+1} s(n, k-1) .$
Given an element $ \sigma$ in $S_{n+1,k} \setminus A,$ we observe that
$\sigma$ has $k$ cycles in its decomposition. So, we can write $l= 
\sigma(n+1).$ Then $\sigma \circ (n+1 \; l)$ is in $S_{n,k}.$
Conversely, if $\sigma \in S_{n,k}$ then for any $l$ in $[n],$
$\sigma \circ (l \, n+1)$ is in $S_{n+1,k}.$
To make this observation more ludic, we can write any element of $ \lvert S_{n+1,k} \setminus A \rvert $ 
as $\sigma = \sigma_1 \sigma_2 \cdots \sigma_k,$ where $\sigma_i$ is 
a cycle for each $i$ in $[k]$ and $\sigma_i, \sigma_j$ are disjoint for
disjoint $i,j$ in $k].$ Let $l = \sigma^{-1}(n+1)$ and $i$ be
the unique element in $[k]$ such that $\sigma_i(l) = n+1.$
This uniqueness follows from the disjointness of the cycles.
So, we may write 
\begin{align*}
	\sigma ={}& \sigma_1 \cdots \sigma_{i-1} (a_1 \, a_2 \, \dotsc l \, n+1) \sigma_{i+1} \cdots \sigma_n \\
	={}& \sigma_1 \cdots \sigma_{i-1} (a_1 \, a_2 \, \dotsc l ) (l \, n+1) \sigma_{i+1} \cdots \sigma_n \\
	={}& \sigma_1 \cdots \sigma_{i-1} (a_1 \, a_2 \, \dotsc l ) \sigma_{i+1} \cdots \sigma_n (l \, n+1).
\end{align*}
Here,  $\sigma_1 \cdots \sigma_{i-1} (a_1 \, a_2 \, \dotsc l ) \sigma_{i+1} \cdots \sigma_n $ is in $S_{n,k}.$
The number of elements in $S_{n+1,k}$ is $-s(n,k)$ and each element in 
$S(n,k)$ corresponds to $n$ distinct elements in $S(n+1,k)$ arising from $n$ distinct choices of the image $\sigma^{-1}(n+1) = l.$
\end{proof}
