
\lecture{Lecture - 13 \hfill 27 Sep 2024, Fri}

Last time, we talked about recurrence relations, generating functions and formal power series.
For the Fibonacci sequence given by
$$ F_0 = 0, \qquad F_1 = 1, \qquad F_n = F_{n-1} + F_{n-2}$$
for all integers $n \geq 2,$
we set $f(t) = \sum_{n \in \mathbb{N}}^{} F_n t^n$ and call it the \emph{Generating Function}
of this sequence.
Solving by substitution for $n=1,2,$ we get
\begin{align*}
	f(t) ={}& \sum_{n=1}^{\infty} F_n t^n \\
	={}& F_0 + t F_1 \sum_{n=3}^{\infty} F_{n-2} t^n + F_{n-2} t^n \\
	={}& t + (t + t^2) \sum_{n=1}^{\infty} F_{n} t^n \\
	={}& t + (t + t^2) f(t)
\end{align*}
Solving for the recurrence relation $f,$ we get
$$ f(t) =  \frac{t}{1 - t - t^2}.$$
Since $ 1 - t - t^2$ has roots 
$$\alpha = \frac{1 + \sqrt{1 + 4}}{2} = \frac{1 + \sqrt{5}}{2}$$ and 
$$\beta = \frac{1 - \sqrt{1 +4 }}{2} = \frac{1 - \sqrt{5}}{2}.$$ 
Using the expression
\begin{align*}
	( 1 - \alpha t)(1 - \beta t) ={}& 1 - (\alpha + \beta) t + \alpha \beta t^2\\
	={}& 1 - t - t^2 
\end{align*}
and the fact $ \frac{1}{\sqrt{5}}(\alpha - \beta) = 1,$ we get
\begin{align*}
	f(t) ={}& \frac{t}{(1 - \alpha t ) ( 1 - \beta t)} \\
	={}& \frac{1}{\sqrt{5}} \left[  \frac{( \alpha - \beta ) t}{
	(1 - \alpha t)(1 - \beta t) }\right] \\
	={}& \frac{1}{ \sqrt{5}} \left[ \frac{1}{1 - \alpha t} - \frac{1}{ 1 - \beta t} \right].
\end{align*}
Recall that $ \frac{1}{1-s} = \sum_{n=1}^{\infty} s ^n$ for any real number $s.$
This allows us to write
\begin{align*}
	\sum_{n=1}^{\infty} F_n t^n ={}& \frac{1}{\sqrt{5}} \left[ \sum_{n=1}^{\infty} 
	(\alpha t)^n - \sum_{n=1}^{\infty} ( \beta t)^n \right] \\
	={}& \sum_{n=1}^{\infty} \frac{1}{\sqrt{5}} \left( \alpha^n - \beta^n \right) t^n.
\end{align*}
Therefore, $F_n = \frac{1}{\sqrt{5}} \left( \alpha^n  -\beta^n \right)$
from comparing the coefficients.

Since $ \lvert  \beta \rvert < 1,$ we get that $ \beta^n \to 0$ as $ n \to \infty.$
So $ \lvert  F_n - \frac{\alpha^n}{\sqrt{5}} \rvert \to 0$ as $n \to \infty.$
In other words $F_n \sim \frac{1}{\sqrt{5}} \alpha^n$ grows exponentially as $ \lvert 
\alpha\rvert >1.$

\section{Notation for approximations}
\begin{definition}[O notation]
	If $f,g$ are functions from $\mathbb{R}_+$ to itself, we say that $f(t) = O(g(t))$ if
	there exists finite real numbers $c, T$ such that $ f(t) \leq c g(t)$ whenever $ t > T.$
\end{definition}

\begin{definition}[O notation for sequences]
	Let $ \{ a_n \; : \; n \in \mathbb{N} \}$ be a sequence,
	and $ f\colon \mathbb{N} \to \mathbb{R}$ be a function.
	We say that $a_n = O(f(n))$ if there exists $ c $ in $\mathbb{R}$ and 
	$\mathbb{N}$ such that $ a_n \leq c f(n)$ for all $n $ in 
	$\mathbb{N}$ whenever $n \geq N.$
\end{definition}
For example, $2^n = O(n!).$
\begin{definition}[Little `o' notation]
	Let $f \colon \mathbb{R}_+ \to \mathbb{R}_+.$
	We say that $f(t) = 0(g(t))$ if $\lim_{t \to \infty} \frac{f(t)}{g(t)} = 0,$
	that is $f(t)$ is growing slower than $g(t).$
\end{definition}
\begin{example}
	We say that $a_n \sim f(n)$ if $\lim_{n \to \infty} \frac{a_n}{f(n)} = 1.$	
\end{example}
Note that $a_n = f(n) + o(f(n)) \Rightarrow a_n \sim f(n),$ if $\lim_{n \to \infty} 
f_n = \infty.$
\begin{theorem}[Stirling's Approximation]
	For any positive integer $n,$
	$$ n! \sim \sqrt{2 \pi n} \left( \frac{n}{e} \right)^n .$$
	In other words
	$$ \ln(n!) = n \ln \left( \frac{n}{e}  \right) - n + \frac{1}{2} \ln(n)
	+ \frac{1}{2} \ln(2 \pi ) + o(1).$$
\end{theorem}

\begin{remark}
	Note that
	\begin{align*}
		n ={}&  o(n \log (n)) \\
		\ln(n)={}& o(n) \\
		\frac{1}{2} \ln(2 \pi) ={}& o(\ln(n))\\
		o(1) ={}& o\left( \frac{1}{2} \ln(2 \pi) \right)
	\end{align*}
	This means 
	$$ \ln(n!) = n \ln(n) + o(n \ln(n)) = (n \ln n)(1 + o(1))$$
	so that $\ln(n!) \sim n \ln n.$
\end{remark}

\begin{proof}[Sketch of proof]
	For any integer $n > 0,$ we have
	\begin{align*}
		\ln(n!) ={}& \sum_{i=1}^{n} \ln i \sim \int_1^n \ln x \mathrm{d} x.
	\end{align*}
\end{proof}

