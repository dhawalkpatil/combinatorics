\lecture{Lecture - 08\hfill 16 Sep 2024, Mon}
\begin{theorem}[Eueler-Hierholzer Theorem]
	A graph $G$ is Eulerian, i.e., has a trail containing each edge exactly once, if and only if it is connected and each vertex has even degree.
\end{theorem}
\begin{proof}
	Assume that $G$ is connected and each of its vertices has even degree. We want to prove that it is Eulerian. We will use induction on the number $ \lvert E \rvert$ of edges in $G.$

	Let $m = 1.$ The only connected graphs having exactly one edge are $ G_1 =  ( \{v_1\}, \{\{v_1,v_1\}\} ),$ the graph with one vertex and a loop on it, and $G_2 = ( \{v_1, v_2\}, \{\{v_1,v_2\}\})$ the graph with two vertices and exactly one edge between them. The latter has vertices with odd degrees, so only consider $G_1.$
$G_1$ is Eulerian because the loop is a trail.

Let us assume that the statement is true for $ \lvert E \rvert = 1, \dotsc, k$ where $k \in \mathbb{N}$ is randomly chosen.
We now take a connected graph $G$ with $k+1$ edges such that every vertex has even degree.
By the preceeding lemma, $G$ contains a cycle $C$ because every vertex has degree $\geq 2.$
Let $G'$ be the graph obtained by removing this cycle form $G.$

Let $\tilde{G}$ be a connected component of $G'.$
Then $\tilde{G}$ is trivially connected, and for all vertices $v$ in $V(\tilde{G}),$
then the degree of a vertex $v$ in the induced subgraph $G'$ is $$ \deg_{G'}(v) =
\begin{cases}
	\deg_G(v) & \text{ if } v \not \in C \\
	\deg_G(v) & \text{ if } v \in C \\
\end{cases}.$$
Thus, each vertex of $\tilde{G}$ is even and it satisfies the induction hypothesis.
It is thus Eulerian.

We proceed to construct an Eulerian walk on $G.$
Let $P = v_0, e_1, v_1, \dotsc, e_s, v_s$ be an Eulerian walk on a collection $ \mathcal{C}$
containing
the cycle $C$ and some or none of the connected components of $G'.$
If there is a vertex $v$ which is not in $P,$ then let $\tilde{G}$ be the connected component 
containing $v.$
There exists some $l \in \mathbb{N}$ such that $v_l \in \tilde{G}$ is contained in $P.$
This is because we assumed $G$ is connected and removed $C$ from it.
So, $v$ must have a path connecting it to $C,$ and it can only contain vertices
from $\tilde{G}$ and from $C$ as any vertex in this path is either in $P$ 
or in the same connected component as $v$ which is $\tilde{G}.$
As $C$ is non empty, $\tilde{G}$ has fewer than $k$ edges and has an Eulerian trail
by the induction hypothesis. Let this trail be 
$ v_l = x_0, e_1, x_1, e_2, x_2, \dotsc, e_r, x_r = v_l$
We then find that 
$  v_0, e_1, v_1, \dotsc, e_l,v_l = x_0, e_1, x_1, e_2, x_2, \dotsc, e_r, x_r = v_l, e_{l+1}, v_{l+1}, \dotsc, 
e_s = v_s = v_0$
is an Eulerian trail on $\mathcal{C} \cup \left\{ \tilde{G} \right\}.$

By induction, we have an Eulerian trail on $G.$

\end{proof}

\begin{definition}[Bipartite graphs]
	A graph $G = (V, E)$ is said to be bipartite if $V = A \cup  B$ with
	$ A \cap B = \phi,$ such that both $A$ and $B$ are independent sets.
	In this case, we call $A$ and $B$ the \emph{partitite} sets of $G.$
\end{definition}

We have the following characterisation of bipartite graphs.



\begin{theorem}[K\"onig]
	A graph $G$ is bipartite if and only if it has no odd cycles.
\end{theorem}

\begin{lemma}
	A graph is bipartite if and only if all its connected components are bipartite.
\end{lemma}

\begin{proof}	
	Given this lemma, we can prove the theorem by assuming that $G$ is connected.
\end{proof}
