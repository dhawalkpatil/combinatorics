\lecture{Lecture - 25\hfill 25 Oct 24, Fri}

\begin{theorem}
	asdfghjkl
\end{theorem}
\begin{proof}
	Let $i,j$ be in $V.$ If $i=j,$ then $L(i,j)$ is the degree
	$\deg(i).$ 
	\begin{align*}
		(N N^T) (i,i)
		={}& \sum_{e \in E} N(i,e) N^T(e,i) \\
		={}& \sum_{e \in E} N(i,e)^2 \\ 
		={}& \sum_{e \in E} \mathbf{1}_{i \in E} \\
		={}& \deg(i).
	\end{align*}
	So, $(NN^T)(i,i) = L(i,i)$ for all $v$ in $V.$
	If otheriwse $i$ is not $j,$ then we have two cases.
	Case 1: $i$ is in $N(j):$ $L(i,j) = -1.$
	If $\tilde{e}$ is $\{i,j\},$ then
	$N(i, \tilde{e}) N^T( \tilde{e}, j) = 
	N(i, \tilde{e}) = N(j, \tilde{e}) = -1.$

	Case 2: $i$ is not in $N(j),$ that is $i$ is not a neighbour
	of $j,$ $L(i,j) = 0.$ So
\begin{equation*}
	(NN^T)(i,j) = \sum_{e \in E} N(i,e) N(j,e)  = 0.
\end{equation*}

To prove the second statement, we observe that $L = D - A$ is real
and symmetric; So $L$ has real characteristic values. Then $L v = 
\lambda v.$ for some characteristic vector $v$ associated with $\lambda.$
In other words, 
\begin{align*}
	(NN^T)v ={}& \lambda v \\
	\lambda \lVert v \rVert^2 = v^T (N N^T v) ={}& v^T (\lambda v)\\
	={}& (v^T N)(N^T v) \\
	={}& (N^T v)^T (N^T v)\\
	={}& \lVert N^T v \rVert^2 \geq 0. 
\end{align*}
Since $v$ is not the zero vector, by definition of characteristic 
vectors, we get that $\lambda$ is non negative.

Now we prove the third statement. We already know that $0$ is a 
characteristic value of $L$ with characteristic vector $\mathbf{1},$
the vector with $1$ in all entries.
We already know that $0$ is a characteristic value of $L,$ and 
$\mathbf{1}$ is a corresponding characteristic vector of it.
We proceed to prove that if $G$ is connected then the geometric
multiplicity of $o$ is 1. As the matrix $L$ is symmetric, it is 
diagonalisable and hence the algebraic multiplicity is the
geometric multiplicity. We have
\begin{align*}
	0 = {}& \frac{\lVert N^T v  \rVert^2}{\lVert v \rVert^2} \\
	={}& \frac{\sum_{ \{i,j \} \in E} \left\{ v(i) - v(j) \right\} ^{2}}
	{\sum_{i \in V} \lVert v \rVert^2} 
\end{align*}
Check that
\begin{align*}
	(N^T)(e) ={}& \sum_{k \in V} N^T(e,k) v(k) \\
	={}& \sum_{k \in V}  N(k,e) v(k) \\
	={}& N(i, e) v(i) + N(j,e) v(j) \\
	={}& v(i) - v(j), \text{ or } v(j) - v(i),
\end{align*}
where $\{i,j\}$ is the representation of the edge $e.$
If $e = \{ i,j\},$ then $N(k,e) = 0,$ unless $k$ is either $i$ or $j.$
Also, $N(i,e)$ and $N(j,e)$ are $1$ and $-1$ in some order.
Therefore, $v(i) = v(j)$ if $\{i,j\}$ forms an edge in $G.$
So, $v(i)$ is constant for all vertices $i$ in a given component
in $G.$ So $v$ is a scalar multiple of $\mathbf{1}.$ So, the algebraic
multiplicity of the characteristic value $0$ is $1.$

Step 2: Suppose $G$ is not connected. We may express the Laplacian $L(G)$
as a block matrix. The diagonal blocks of $L(G)$ would be the Laplacian
of the corresponding to the components of $G.$

\end{proof}


HOW TO OPEN LATEX ERROR LOG FILE IN VIMTEX?
