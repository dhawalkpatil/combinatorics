\lecture{Lecture - 19\hfill 11 Oct 24, Fri}

\begin{definition}[The falling factorial]
	For any positive integer $n,$ we wdefine the \emph{falling factorial} to be 
	$$ (t)_n = t(t-1) \cdots (t - n+1).$$
\end{definition}


\begin{definition}[The rising factorial]
	For any positive integer $n,$ we define the \emph{rising factorial} to be 
	$$ t^{(n)} = t(t+1) \cdots (t + n-1).$$
\end{definition}


\begin{remark}
	$$\binom{ \alpha } {n} = \frac{( \alpha)_n}{n!} $$
\end{remark}

\begin{remark}
	$ t^{(n)} = (-1)^n (-t)_n $
\end{remark}


\begin{theorem}
	For any $t$ in $\mathbb{R}$ and any $n $ in $\mathbb{N},$ we have
	$$(t)_n = \sum_{k \in [n]} s(n,k) t^k. $$
\end{theorem}

\begin{remark}
	For fixed $n$ in $\mathbb{N},$ the generating function of 
	$\left\{ s_{n,k} \;:\; k \in \mathbb{N} , k \leq n \right\} $ is
	$$ (t)_n = \sum_{k \in [n]} s(n,k)t^k. $$
\end{remark}
\begin{proof}
	For $n=1,$ $s(1,1) = 
	\begin{bmatrix} 1\\1 \end{bmatrix}.$
Suppose the formula is true for some $n$ in $\mathbb{N}.$ Then,
\begin{align*}
	(t)_{n+1}
	={}& (t-n) (t)_n \\
	={}& (t-n) \left( \sum_{k \in [n]} s(n,k) t^k \right)  \\
	={}& \sum_{k \in [n]} s(n,k) t^{k+1} - n \sum_{k \in [n]} s(n,k) t^k  \\
	={}& \sum_{k=0}^{n-1} s(n,k-1) t^k - \sum_{k=0}^{n} n s(n,k) t^k\\
	={}& 0 + \sum_{k=1}^{n-1}  s(n,k-1) t^k - \sum_{k=0}^{n-1} n 
	s(n,k) t^k - n s(n,n) \\
	={}& \sum_{k=1}^{n-1} s(n+1,k) t^k + (-1)^{n-k+1} (n-1)! \times n t^{n+1}\\
	={}& \sum_{k=0}{n} s(n+1,k) t^k.
\end{align*}
\end{proof}



\section{Stirling Numbers of the Second Kind}

\begin{definition}
	The Stirling numbers of the second kind are denoted as 
	$$ \begin{Bmatrix} n\\k \end{Bmatrix} = S(n,k)$$
	for positive integers $n,k$ and they represent the
	number of partitions of $[n]$ into $k$ nonempty blocks.
\end{definition}


\begin{remark}
	As the $n^\text{th}$ Bell number $B_n$ is the number of partitions is the toal number of partitions of $[n]$ into nonempty blocks, we have
	$$B_n = \sum_{k=1}^{n} \begin{Bmatrix} b\\k\end{Bmatrix}.$$ 
\end{remark}



Any partition of $[n]$ into $k$ nonempty blocks is related to the
surjective functions from $[n]$ to $[k].$ Given a surjective function
$f \colon [n] \to [k],$ the sets $\{ f^{-1}(1), \dotsc, f^{-1}(k)\}$
form a partition of $[n]$ into $k$ non empty blocks.

Conversely, given a partition $A_1, A_2, \dotsc, A_k$ of $[n]$ into $k$ nonempty blocks and $\sigma$ in $S_k,$ $f \colon [n] \to [k]$ defined by
$f(i) = \sigma(j)$ such that $i$ is in $A_{j}$ is a surjective function.
Thus, we have proved the following theorem
\begin{theorem}
	If $n \geq k,$ then the number of surjective maps from
	$[n]$ to $[k]$ is expressed by the following relation
	\begin{align*}
		k! \begin{Bmatrix} n\\k\end{Bmatrix}
		={}& \lvert \{ f\colon [n] \to [k] \; | \;
		f \text{ is surjective } \} \rvert \\
		={}& \sum_{j=0}^{k} (-1)^j \binom{k}{j} (k-j)^n.
	\end{align*}
\end{theorem}


\begin{remark}
	We have the recurrence relation for positive integers $k \leq n,$
	$$ \begin{Bmatrix} n\\k\end{Bmatrix} = 
	\begin{Bmatrix}n-1\\k-1 \end{Bmatrix} + 
	k \begin{Bmatrix}n-1\\k-1 \end{Bmatrix}. $$
\end{remark}


\begin{theorem}
for any $t$ in $\mathbb{R},$ and any $n$ in $\mathbb{N},$ we have
$$	 t^n = \sum_{k \in [n]} 
\begin{Bmatrix} n\\k \end{Bmatrix} (t)_k. $$
\end{theorem}

\begin{proof}
	Assuming that $$t^{k} = \sum_{i \in [k]} 
	\begin{Bmatrix} n \\ i \end{Bmatrix} (t)_i,$$
\begin{align*}
	t^{k+1} ={}& (t-k) t^k + k t^k \\
	={}& COMPLETE THIS!
\end{align*}
\end{proof}

\begin{definition}
	A vector space $V$ over a field $\mathbb{F},$ is a set
	$A$ equipped with two operations $+ \colon A \times A \to V$ usually called `addition' and
	$\ \cdot \  \colon \mathbb{F} \times A \to A$ called the `scalar multiplication' such that for any $\alpha, \beta$ in
	$\mathbb{F}$ and $x,y$ and $z$ in $A,$
	\begin{enumerate}
		\item $\alpha(\beta x) = (\alpha \beta)x,$
		\item $(\alpha + \beta) x = \alpha x + \beta x,$
		\item $\alpha(x + y) = \alpha x + \alpha y,$
		\item $(\alpha + \beta) x = \alpha x + \beta x,$
		\item $ 1 x = x,$
		\item $x+y = y+x,$
		\item $(x+y)+z = x + (y+z),$
	\end{enumerate}
	where $1$ and $0$ are the multiplicative and additive identity of $\mathbb{F}$ and there exists $\mathbf{0}$ in $A$ such that $0 x = \mathbf{0}$
	for all $x$ in $A.$
\end{definition}

Hereafter, for a vector space $V,$ the underlying set $A$ will also be 
represented by $V.$

The vector space of polynomials over an unkown $t$ with coefficients
in $\mathbb{F},$ is a vector space over $\mathbb{F}.$

Let $\mathcal{P}_n$ be the (sub)space of all polynomials in
$\mathbb{F}[t]$ of degree at most $n.$ Then  
the dimension of $\mathcal{P}_n$ is $n+1$ because $\{ 1, t, t^2,
\dotsc, t^n \}$ forms a basis of it.

Note that $\{ (t)_0, (t)_1, \dotsc, (t)_n \}$ and 
$\{ t^{(0)}, t^{(1)}, \dotsc, t^{(n)} \}$ form two separate bases of 
$\mathcal{P}_n.$ The change of basis formulae for these bases are given
by the relation between $(t)_i$ and  $t^{(i)}.$

If $A = \{ s(i,j) \}$ and $ B = \{S(i,j) \} $ are $n \, \times  \, n$
matrices, then $A = B^{-1}.$

The convention is to write 
$$\begin{bmatrix}n\\0\end{bmatrix} =
\begin{Bmatrix}n\\0\end{Bmatrix} = 
\begin{cases}
0 & \text{ if } n \in \mathbb{N} \\
1 & \text{ if } n = 0 \\
\end{cases}. $$

