\lecture{Lecture - 23 \hfill 21 Oct 24, Mon}

\begin{theorem}
	If $ \lambda_1,$ $ \lambda_2, \dotsc, \lambda_n$ are characterstic
	values of  
	such that $ \lambda_1 \geq \lambda_2 \geq \cdots \geq \lambda_n.$
	The following statements are equivalent
	
	\begin{enumerate}
		\item $G$ is bipartite.
		\item The spectrum of $G$ is symmetric about $0.$
		\item If $n$ is even, the characterstic polynomial $\det(t I_n - a)$ of $A$ is a polynomial in $t^2.$
	\end{enumerate}
\end{theorem}

\begin{proof}
	$(1) \implies (2)$ Was shown in the previous lecture.	

	$(2) \implies (3)$ If $n$ is even, then the multiplicity of $0$
	as a characteristic value is even (or $0$) because the nonzero
	characteristic values appear in pairs. As the adjacency matrix
	is symmetric, it has $n$ characteristic values and hence the
	characteristic polynomial $\det(t I_n - A) = \prod_{j=1}^{n} (t - \lambda_i)$
	is of the form $t^{2j} \prod_{ \lambda \in \operatorname{spec}(G) \setminus\{0\} } (t^2 - \lambda^2) .$

	$(3) \implies (2)$  We are given that
	$\det(t I_n - A)$ is of the form $t^l f(t)$ where
	$f(t)$ is a polynomial in $t^2.$ It can thus be 
	factorised into a product of the form 
	$ t^l \prod_{j=1}^{n/2 - l} (t^2 - a_j)$
	where $l$ is the mulitiplicity of $0$ as a characteristic value.
	Here, $a_j <0$ for each $j$ because all characteristic values
	of $A$ are real.

	$(2) \implies (4)$
	COMPLETE THIS!!

	$(4) \implies (1)$ 
	Recall that $\sum_{i=1}^{n} \lambda_i^l$ is the number of closed
	walks in $G$ of length $l.$ As $\det(t I_n - A)$ is a polynomial
	in $t^2,$ we get that $\sum_{i=1}^{n} \lambda_i^l$ is zero for
	all $l.$ So, $G$ has no closed walks of odd length. As we had 
	characterised that a graph is bipartite if and only if it has no
	closed walks of odd length, we realise that $G$ is bipartite.
\end{proof}

\section{Using the Spectrum to estimate the diameter}
For a subset $A$ of $\mathbb{R}^n,$ or a any metric space $(S,\rho),$
we define the diameter of $A$ by
$$ \operatorname{diam}(A) = \sup_{x,y \in A} \rho(x,y). $$
Similarly,
\begin{definition}
	For a connected graph $G,$ we define the diameter of $G$ is 
	$$\operatorname{diam}(G) = \max_{x,y \in V} d(x,y),$$
	where $d(x,y)$ is the minimum distance between vertices $x$ and
	$y$ given my $\min\{ \lvert w \rvert \, : \, w \text{ is a walk
	from } x \text{ to } y \}.$
\end{definition}

\begin{theorem}
	If $G$ is a connected graph whose adjacency matrix has $r$ 
	distinct characteristic values, then $\operatorname{diam}(G)
	\leq r-1.$
\end{theorem}

\begin{remark}
	If some characteristic values have a (relatively) high
	multiplicity, then there are only a few distinct characteristic
	values which means $\operatorname{diam}(G)$ is small.
\end{remark}
For a matrix $A,$ in $A(n,n \mathbb{R}),$ the minimal
polynomial of $A$ is the monic polynomial $m$ in $\mathbb{R}[t]$
of smallest degree such that $m(A) = 0.$
Note that $m(t) = \prod_{i \in [r]} (t - \Theta_i),$  where
$\Theta_1, \dotsc, \Theta_r$ are the distinct characteristic values
of $A,$ if and only if $A$ is diagonalisable, as it is if $A$ is
symmetric.
\begin{definition}
	The \emph{Adjacency Algebra} of the graph $G$ is 
	$$ \mathcal{A}(G) = \left\{ p(A) \, : \, p \in \mathbb{R}[t] 
	\right\} $$
	where $A$ is the adjacency matrix of $G.$
\end{definition}

\begin{remark}
	$\mathcal{A}(G)$ is a subset of the vector space of
	$n \times n$ symmetric matrices. In fact, it is a linear 
	subspace of the space of $n \times n$ symmetric matrices.
	In fact, it is a linear subspace, moreoever an algebra.
\end{remark}

If $A$ is a diagonalisable matrix with $r$ distinct characteristic 
values, then the dimension of $\mathcal{A}(G)$ as a vector space is $r.$
\begin{proof}
	Let $d$ be the diameter of $G.$ We want to show that
	$I_n, A, A^2, \dots, A^d$ are linearly independent matrices.
	This will prove the theorem.
	Given a linear combination
	$\alpha_0 I_n + \alpha_1 A + \cdots + \alpha_k A^k,$
	with $k \leq d,$
	there exist vertices $x$ and $y$ in $V$ with $d(x,y) = k.$
	Such a pair of vertices must exist because there is a path of length $d$ in $G.$
	Any two vertices $x,$ $y$ in any path separated by $k$ edges
	must satisfy $d(x,y) = k.$
	In other words, there does not exist a shorter path
	from $x$ to $y.$ So, we get that the $(x,y)^\text{th}$ entry 
	of the linear combination
	is $$ \alpha_0 I_n(x,y) + \alpha_1 A(x,y) + \alpha_2 A^2(x,y)
	+ \cdots + \alpha_k A^k(x,y).$$
	Here $A^i(x,y)$ is the number of walks of length $i$ from $x$
	to $y$ for each $i.$ As there are no walks of length less than
	$d$ from $x$ to $y,$ we must have $A(x,y),$ $A^2(x,y), \cdots,
	A^{k-1}(x,y)$ are all zero. Also, $x,$ $y$ are distinct because
	they are two vertices separated by $k$ edges on a path. So,
	$I_n(x,y) = 0.$ Thus, $\alpha_k A^k(x,y)$ is zero.
\end{proof}

