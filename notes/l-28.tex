\lecture{Lecture - 28\hfill 04 Nov 24, Mon}

Proving Cayley's theorem
We want to show that the number of (labelled) trees on $n$ vertices is $n^{n-2}.$
For $m,n$ in $\mathbb{Z}_+,$ let $A$ be the set of all connected graphs
on $[n]$ with $m$ edges and let $G(n,m) = \lvert A \rvert.$
Let $F(n,m) $ be the number of graphs in $A$ with no leaves.
For $j$ in $[n],$ let 
\begin{align*}
A_j =& \left\{ G \in A \;:\; \deg_G(j) = 1\right\} \\
=& \left\{ G \in A \;:\; j \text{ is a leaf in } G\right\}.
\end{align*}

So, $F(n,m) = \lvert A \setminus \bigcup_\{j \in [n]\} A_j \rvert.$
Note that $\lvert A_j \rvert = G(n-1, m-1) (n-1).$
More generally if $J \subset [n],$ and $A_J = bigcap_{j \in J} A_j,$
$$\lvert A_J = G(n - \lvert J \rvert, m - \lvert J \rvert) ( n-1)^{\lvert
J \rvert}.$$
By the Inclusion-Exclusion Principle,
$$F(n,m) = \sum_{J \subset [n]} (-1)^{\lvert J \rvert} G(n- \lvert J
\rvert, m - \lvert J \rvert) $$
...
The number of connected graphs on $[n]$ with $m$ edges and no leaves is
$$F(n,m) = \sum_{i=0}^n (-1)^i \binom{n}{i} G(n-i, m-i) G(n-i)^i.$$
The characterisation of trees implied that any connected graph on 
$[n]$ with $n-1$ edges is a tree.
So, $G(n,n-1) $ is the number of trees on $[n].$
As $F(n, n-1)$ is the number of trees on $[n]$ with no leaves.
So we conclude the
\begin{lemma}If $T_n$ is the number of trees on $[n],$ then
$$\sum_{i=0}^n (-1)^i \binom{n}{i} T_{n-i} (n-i)^i = 0.$$
\end{lemma}
We want to solve this equation for $\{T_n\}$ in the form of an explicit
formula.
We use induction to show that $T_n = n^{n-2}.$
For $n=2,$ $T_2 = 1 = 2^{2-2}.$ Suppose this is true for $i = 1,2, \dotsc,
k.$ We proceed to show that is holds true for $i=k+1.$
Using the preceeding lemma, we get
\begin{align*}
(-1)^0 \binom{k+1}{0} T_{k+1} =& - \sum_{i=1}^{k+1} (-1)^i \binom{k+1}{i}
T_{k+1-i} (k+1-i)^i \\
=& - \sum_{i=1}^{k+1} (-1)^i \binom{k+1}{i} (k+1-i)^{k-1-i} (k+1-i)^i \\
=& -\sum_{i=1}^{k+1} \sum_{j=0}^{k-1} (-1)^i 
\binom{k+1}{i} \binom{k+1}{j} (k+1)^j i^{k+1-j} \\
\end{align*}

\section{Spanning trees}
Suppose we are given a graph $G = (V,E).$
\begin{definition}
A spanning subgraph of $G$ is a graph on the vertex set $V$ whose edge set is a subset of $E.$
A spanning tree in $G$ is a spanning subgraph which is a tree.
