\lecture{Lecture 16 \hfill 04 Oct 204, Fri}

\begin{definition}[Double Factorial]
	For any odd integer $n,$ we define the \emph{double factorial}
	$n!!$ of $n=2k-1$ by $1 \times 3 \times  5 \times \cdots 
	\times 2k-1.$
\end{definition}
For $n = 2k-1,$ we have
$$ n!! = \frac{(2k)!}{k! 2^k}.$$


\begin{lemma}
	For any positive integer $n,$ we have
	$$ \binom{1/2}{n+1} = 
	\frac{(-1)^n}{ (n+1) 2^{2n+1}} \binom{2n}{n}.$$
\end{lemma}

\begin{proof}
	Given a positive integer $n,$ we have
	\begin{align*}
		\binom{1/2}{n+1}
		={}& \frac{1/2 (1/2-1)\cdots(1/2-n)}{(n+1)!}\\
		={}& \frac{(1-2)(1-4)\cdots(1-2n)}{2^{n+1} (n+1)!} \\
		={}& \frac{(-1)^n}{2^{n+1} (n+1)!} (2n-1)!!\\
		={}& \frac{(-1)^n}{2^{n+1} (n+1)!} \frac{(2n)!}{2^n (n!}\\
		={}& \frac{(-1)^n}{(n+1) 2^{2n+1}} \frac{(2n)!}{(n!)^2}
	\end{align*}
	
\end{proof}


Using this expression for $\binom{1/2}{n},$ we can rephrase the 
expression for Catalan numbers in the following way.
\begin{align*}
	C_n ={}& \frac{\binom{1/2}{n+1} (-4)^{n+1}}{2} \\
	={}& - \frac{1}{2} \frac{(-1)^n}{(n+1) 2^{2n+1}} \binom{2n}{n}
	(-4)^{n+1} \\
	={}& \frac{1}{n+1} \binom{2n}{n} .\\
\end{align*}

\section{How does $C_n$ behave as $n$ gets large?}
From Stirling's Formula we know that
$n! \sim \sqrt{2 \pi n} \left( \frac{n}{e} \right) ^n.$ We can deduce
$$ \binom{2n }{n} = 
\frac{(2n)!}{(n!)^2} \sim
\frac{\sqrt{2\pi 2n} \left( \frac{2n}{e}^{2n} \right) }{ 2\pi n (
\frac{n}{e})^{2n}}.$$
which implies that
$\binom{2n}{n} \sim \frac{1}{\sqrt{\pi n}} 2^{2n}.$
Therefore
$$ C_n = \frac{1}{(n+1)} \binom{2n}{n}
\sim \frac{4^n}{(n+1) \sqrt{\pi n}}
\sim \frac{4^n}{ \sqrt{\pi} n^{3/2}}.$$
In other words,
$\log_2 C_n = 2n - \frac{3}{2} \log_2 n - \frac{1}{2} \log_2 \pi + o(1).$

\section{Aside: Another use for Catalan Numbers}
For any positive integer $n,$ $C_n$ also represents the number of ways
of breaking down the sum of $n+1$ numbers into pairwise operations.

For instance, when $n=2,$
$(x_1+  x_2) + x_3 = x_1  +(x_2 + x_3),$ and
when $n=3,$ 
$$(x_1 + x_2) + (x_3 + x_4) = (x_1 + (x_2 + x_3)) + x_4 $$
$$ = ((x_1 + x_2) + x_3) + x_4 = x_1 + (x_2 + (x_3 + x_4))$$
$$ = x_1 + (x_2 + x_3 ) +x_4 .$$

\section{Bell Numbers}
\begin{definition}
The $n^\text{th}$ \emph{Bell number $B_n$} is trhe number of partitions
of a set of $n$ elements.
\end{definition}

Recall that a partition of a set $A$ is a collection of nonempty
subsets $ \{ A_1, \dotsc, A_k\}$ such that 
$A_i \cap A_j =  \phi$ if $i \not = j$ and $\bigcup_{i \in [k]} A_i = A.$

\begin{example}
	$B_1 = 1$ because the only partition of $[1]$ is $\{\{1\}\}.$

	$B_2 = 2$ because the only partitions of $[2]$ are
	$\{ \{ 1,2\} \}$ and $\{ \{1\}, \{2\} \}.$

	$B_3 = 5$ because $\{ \{ 1\} , \{ 2, 3\} \},$ 
	$\{ \{1 \}, \{2\}, \{3\}\},$ $\{ \{ 1,2,3\}\},$
	$\{ \{1, 2\}, \{3\}\}$ and $\{ \{1,3\}, \{2\}\}$
	are the only partitions of $[3].$
\end{example}

\begin{lemma}
	For $n$ in $\mathbb{N},$
	$$ B_n = \sum_{k = 1}^{n} \binom{n-1}{k-1} B_{n-k}.$$
\end{lemma}

\begin{proof}
	In order for the formula to hold for $n=2,$ we need that
	$2 = B_2 = \sum_{k=1}^{2} \binom{1}{k} B_k
	= \binom{1}{0} B_0 + \binom{1}{1} B_1
	= B_0 + B_1.$
	This means we need to set $B_0 = 1.$
	Suppose $ \{ A_1, A_2, \dotsc, A_k \}$ is a partition of $[n].$
	Then there is exactly one block, say $A_j$ which contains $n.$
	Then, $A_j = \{n\} \cup Y$ for some $ Y \subset [n-1].$
	If $ \lvert A_j \rvert = m,$ then $ \lvert Y \rvert = m-1;$ 
	so $Y$ could be one of $\binom{n-1}{m-1}$ subsets of $[n-1].$
	The remaining elements in $[n] \setminus A_j$ could be 
	partitioned in $B_{n-m}$ ways.
	Therefore,
	\begin{align*}
		B_n ={}&  \sum_{m=1}^{n} \binom{n-1}{m-1} B_{n-m}.
	\end{align*}
\end{proof}
This recurrence relation which we have obtained for $B_n$ is not 
a linear recurrence relation because the number of terms is not fixed
but grows with $n.$
\begin{definition}[Exponential generating function]
	Given a sequence $ \{ a_n\}_n,$ the \emph{exponential
	generating function} of $\{a_n\}_{n \in \mathbb{N}}$ is 
	$$ G(t) = \sum_{n1}^{\infty} \frac{a_n}{n!} t^n $$
\end{definition}
Consider the epxonential generating function of $\{B_n\}.$
We have
\begin{align*}
	G(t) ={}& \sum_{n=0}^{\infty} \frac{B_n}{n!} t^n 
	\intertext{Differentiating it with respect to $t,$ we get}
	G'(t) ={}&  \sum_{n=1}^{\infty} \frac{B_n}{n!} n t^{n-1}\\
	={}& \sum_{n=1}^{\infty} \frac{B_n}{(n-1)!} t^{n-1} \\
	={}& \sum_{n=1}^{\infty} \frac{t^{n-1}}{(n-1)!}
	\sum_{k=1}^{n} \binom{n-1}{k-1} B_{n-k} \\
	={}& \sum_{n=1}^{\infty} \sum_{k=1}^{n} 
	\frac{t^{k-1}}{(k-1)!} \frac{B_{n-k} t^{n-k}}{(n-k)!} \\
	={}& \sum_{n=1}^{\infty} \sum_{k=1}^{\infty} 
	\frac{t^{k-1}}{(k-1)!} \frac{B_{n-k} t^{n-k}}{(n-k)!}
	\chi_{[1,n]}(k) \\
	={}& \sum_{k=1}^{\infty} \sum_{n=1}^{\infty}
	\frac{t^{k-1}}{(k-1)!} \frac{B_{n-k} t^{n-k}}{(n-k)!}
	\chi_{[1,n]}(k) \\
	={}& \sum_{k=1}^{\infty} \frac{t^{k-1}}{(k-1)!}
	\sum_{n=k}^{\infty} \frac{t^{n-k}}{(n-k)!} B_{n-k} \\
	\intertext{Replacing $n$ with $n+k$, we get}
	={}& \sum_{k=0}^{\infty} \frac{t^k}{k!}
	\sum_{n=0}^{\infty} \frac{t^n}{n!} B_n\\
	={}& \left( \sum_{k=0}^{\infty} \frac{t^k}{k!} \right) 
	\left( \sum_{n=0}^{\infty} \frac{t^n}{n!} B_n \right) \\
	={}& e^t G(t).
\end{align*}
Therefore, $G'(t) = e^t G(t)$ which means
$$[\log G(t)] ' = \frac{G'(t)}{G(t)} = e^t.$$
So, $$\log G(t) = e^t + k,$$
where $k$ is some constant in $\mathbb{R}.$
This means $G(t) = A e^{e^t},$ where $A$ is some positive number.
We had that $G(0) = B(0) = 1.$  So, $A e^1 = 1$
and $A = \frac{1}{e}.$
This means $$G(t) = e^{e^t - 1}.$$

