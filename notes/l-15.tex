
\noindent
\emph{Lecture - 15 \hfill 02 Oct 24, Wed}

\begin{definition}[Catalan numbers]
	For any positige integer $n,$ the $n^\text{th}$ Catalan number
	denoted by $C_n$ is the number of ways of triangulating an
	$n$ sided polygon.
\end{definition}

\begin{figure}[h]
	\centering
	\incfig{triangulation-hexagon}{0.4}
	\caption{Triangulation of a hexagon}
	\label{fig:triangulation-hexagon}
\end{figure}

A triangulation of a polygon is a way of expressing the polygon as
a union of non self intersecting triangles.

The recurrence relation for Catalan numbers $C_n$ is 
$$C_{n+1} = \sum_{k=0}^{n} C_k C_{n-k}.$$
\begin{proof}
	Consider any triangulation of an $n+1$ sided polygon and fix one edge of it.
	
\end{proof}

This is a nonlinear recurrence relation. To solve it, set
$$ f(t) = \sum_{n=1}^{\infty} C_n t^n $$
to be the generating function of the Catalan numbers.
Note 
\begin{align*}
	\sum_{n=1}^{\infty} C_{n+1} t^n
	 ={}& \sum_{n=1}^{\infty} C_n t^{n-1} \\
	 ={}& \frac{1}{t} \left[ \sum_{n=0}^{\infty} C_n t^n - C_0 \right] \\
	 ={}& \frac{1}{t} \left[ F(t) - 1 \right].
\end{align*}
 Also,
 \begin{align*}
	 \sum_{n=1}^{\infty} C_{n+1} t^n
	 ={}& \sum_{k=0}^{\infty} i\left( C_k C_{n-k} \right) t^n \\
	 ={}& \sum_{n=1}^{\infty} \left( \sum_{k=0}^{n} C_k t^k C_{n-k}
	 t^{n-k}\right) \\
	 ={}& \sum_{n=1}^{\infty} \sum_{k=1}^{\infty} 
	 C_k t^k C_{n-k} t^{n-k} \chi_{[1,n]} (k) \\
	 ={}& \sum_{k=1}^{\infty} \sum_{n=1}^{\infty} 
	 C_k t^k C_{n-k} t^{n-k} \chi_{[1,n]} (k) \\
	 ={}& \sum_{k=1}^{\infty} C_k t^k
	 \left[ \sum_{n=1}^{\infty} C_{n-k} t^{n-k} \chi_{[1,n](k)} \right] \\
	 ={}& \sum_{k=1}^{\infty} C_k t^{k} \left[ 
	 \sum_{n=k}^{\infty} C_{n-k} t^{n-k} \right] \\
	 ={}& \sum_{k=1}^{\infty} C_k t^k
	 \left[ \sum_{n=0}^{\infty} C_n t^n \right]
 \end{align*}
 Multiplying the entire equation by $t,$ we get
 \begin{align*}
	 F(t) - 1
	 ={}& \sum_{n=2}^{\infty} C_{n} t^n\\
	 ={}& t \sum_{n=1}^{\infty} C_{n+1}t^n\\
	 ={}& t \sum_{k=1}^{\infty} C_n t^n \sum_{n=1}^{\infty} C_n t^n
	 \\
	 ={}& t F(t)^2
 \end{align*}

 This means $t F(t)^2 - F(t) +1 = 0.$
 This is a quadratic equation for $F(t).$
 Solving it gives us
 $$ F(t) = \frac{1 \pm \sqrt{1 - 4t}}{2t}. $$
 As we have $$ \sqrt{1 - 4t} =
 \sum_{n=1}^{\infty} \binom{1/2}{n} (-4t)^n
 = 1 - 2t - 2t^2 - 4t^3 - 10 t^4 - \cdots,
 $$ where $$ \binom{1/2}{n} = \frac{ \frac{1}{2} ( \frac{1}{2} - 1 )
\cdots ( \frac{1}{2} -n + 1)}{n!}.$$
Supopse 
$$ F(t) = \frac{1 - \sqrt{1 - 4t}}{2t}
 = \frac{1}{t} - 1 - t - 2t^2 - 5 t^3 - \cdots$$
Due to the $1/t$term, this is not a power series.
So we must have
\begin{align*}
	F(t) ={}& \frac{1 - \sqrt{4 - t}}{2t}\\
	={}& \frac{1 - 1 \sum_{n=1}^{\infty} \binom{1/2}{n} (-4t)^2}
	{2t} \\
	={}& \sum_{n=1}^{\infty}  - \frac{\binom{1/2}{n+1} (-4t)^{n+1}}{2} t^n.
\end{align*}

By comparing the terms of the power series, we get that
$$C_n = - \frac{\binom{1/2}{n+1} (-4)^{n+1} }{2} $$
